\section{Minispec Tools}
\label{sec:tools}

Minispec includes an easy-to-use toolset to simulate and synthesize circuits.
We first describe its command-line interface, than its Jupyter interface.

\subsection{Command-line tools}
\label{sec:cli}

\paragraph{\texttt{ms}} is Minispec's high-level command-line interface. \texttt{ms} provides four commands:
\begin{compactitem}
\item \verb|ms eval [<file>] <expression>                  | Evaluate expression
\item \verb|ms sim <file> <module>                         | Simulate module
\item \verb|ms synth <file> <function/module> <synthArgs>  | Synthesize function or module into gates
\item \verb|ms help                                        | Print help message
\end{compactitem}
\begin{wrapfigure}{r}{0.35\columnwidth}
\vspace{-1.5em}
\begin{minted}[bgcolor=bashbg]{bash}
# Evaluate "2 + 3"
ms eval "2 + 3"
# Evaluate add#(4)(2, 3), where
# add is defined in file add.ms
ms eval add.ms "add#(4)(2, 3)"

# Simulate module TestCounter
# in file counter.ms
ms sim counter.ms TestCounter

# Synthesize function add#(4)
ms synth add.ms "add#(4)"
# Synthesize module Counter
ms synth counter.ms Counter
\end{minted}
\vspace{-4em}
\end{wrapfigure}

The \verb|<file>| argument should be a Minispec source file with the target function or module.
The file argument is optional for \verb|ms eval|, as it is not needed if the expression does not
call any user-defined functions.
Arguments should be quoted as needed to avoid being interpreted by the shell.

For more details, run \verb|ms help|.

Internally, \verb|ms| uses two lower-level tools:
\textbf{\texttt{msc}}, the Minispec compiler, which compiles functions or modules into BSV, Verilog, or a simulation executable;
and \textbf{\texttt{synth}}, a synthesis tool for Minispec and Bluespec circuits that leverages
the \href{http://www.clifford.at/yosys/}{yosys} open-source synthesis suite
and the \href{https://www.eda.ncsu.edu/wiki/FreePDK45:Contents}{FreePDK45} open-source 45nm standard cell library
to produce optimized gate-level circuit implementations, and to analyze their delay and area.
These tools can be used directly; run \texttt{msc -h}
or \texttt{synth -h} for usage instructions.

\subsection{Jupyter interface} 

Minispec is integrated with \href{https://jupyter.org/}{Jupyter} notebooks.
Minispec Jupyter notebooks follow the usual conventions for notebooks in other languages:
Minispec code can be spread across multiple \emph{code cells},
which the notebook user can execute individually or in sequence.
A code cell can use or redefine functions, modules, types, or constants defined in previously executed cells.
Redefined functions or modules can be used by later code,
but previously executed cells still use the old definitions.
In a Minispec notebook, ``executing'' a code cell only checks the code for errors.
To provide additional capabilities, %for evaluating, simulating, and synthesizing circuits,
cells can include additional built-in commands, called \emph{magics} in Jupyter terminology.

\paragraph{Magics:} Minispec notebooks provide four magics:
\texttt{\%\%eval}, \texttt{\%\%sim}, \texttt{\%\%synth}, and \texttt{\%\%help}.
Each of these magics has the same purpose as the \texttt{ms} command with the same name (\autoref{sec:cli}),
and follows a similar syntax.

There are two syntax differences between Jupyter magics and \texttt{ms} commands.
First, Jupyter magics do not take a file argument;
instead, they work on the code from executed cells.
Second, Jupyter magics do not need quotes on expressions
or parametric functions or modules, as they are not interpreted by the shell.

The \href{https://6004.mit.edu/web/fall19/resources}{Jupyter notebook tutorials} on combinational and sequential logic
serve as detailed examples on using Jupyter notebooks and the different magics.

