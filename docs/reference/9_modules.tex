\section{Modules and Sequential Logic}
\label{sec:modules}

Modules implement sequential logic, i.e., digital logic with \emph{state}.
Specifically, they implement single-clock synchronous sequential circuits,
where state is maintained in \emph{registers} that all share the same periodic clock signal.
Registers update their contents simultaneously, at the rising edge of the clock.
This allows discretizing time into \emph{cycles} and abstracting sequential circuits
as \emph{finite state machines} (FSMs).
As shown in \autoref{fig:fsm}, on each cycle, the FSM stores a particular \emph{state} (in registers)
and takes some \emph{inputs}.
Combinational logic within the FSM uses the state and inputs to compute the
\emph{next state} of the FSM and its \emph{outputs} for the current cycle.
At the end of the cycle (i.e., in the next rising clock edge),
register update their values, placing the FSM into the next state.

Minispec modules have four key elements that correspond to the components of FSMs:
\begin{compactenum}
\item \emph{Submodules}, which can be registers or other user-defined modules to allow composition of modules.
\item \emph{Methods}, which implement combinational logic to produce \emph{outputs} given some input \emph{arguments} and the current state.
\item \emph{Rules}, which implement combinational logic to produce the \emph{next state} given some external \emph{inputs} and the current state.
\item \emph{Inputs}, which represent external signals controlled by the enclosing module.
\end{compactenum}

\begin{figure}[h]
\centering
\begin{minipage}[b]{0.32\columnwidth}
\centering
\includegraphics[height=1.6in]{figures/ag_fsm.pdf}
\caption{Finite State Machine.}
\label{fig:fsm}
\end{minipage}
\hfill
\begin{minipage}[b]{0.32\columnwidth}
\centering
\includegraphics[height=1.6in]{figures/ag_basicmodule.pdf}
\caption{Basic module.}
\label{fig:basicmodule}
\end{minipage}
\hfill
\begin{minipage}[b]{0.32\columnwidth}
\centering
\includegraphics[height=1.6in]{figures/ag_hiermodule.pdf}
\caption{General module.}
\label{fig:hiermodule}
\end{minipage}
\end{figure}


\autoref{fig:basicmodule} shows the structure of a basic module with registers but no other submodules.
Comparing \autoref{fig:fsm} and \autoref{fig:basicmodule}, the key difference
is the distinction between inputs and method arguments: whereas combinational logic in FSMs
can use any input to produce both the next state and the output,
module methods use \emph{arguments} to produce outputs, separate from the \emph{inputs} used by rules.
As we will see, this separation \emph{makes hierarchical composition easy} (\autoref{sec:basicmodsemantics}).
\autoref{fig:hiermodule} shows a general module that includes submodules other than registers.
The separation between rule inputs and method arguments
allows modules to instantiate and use arbitrary submodules \emph{while avoiding combinational cycles},
i.e., ensuring that combinational logic remains acyclic.

% TODO(dsm): I've gotten feedback that this doesn't explain why, but I don't think this explanation belongs here. This is the intro, and the paragraph says "as we will see"; this expla should probably be in 9.3.
%To see why this is the case, consider a different approach where we only had inputs
%and a single block of combinational logic produced both the outputs and the inputs for submodules.
%Without knowing the implementation details of its submodules, a module could wire up submodule inputs and outputs
%in a way that created a combinational cycle, e.g., by setting an input to a submodule
%to a value that depends on one of its outputs.
%By contrast, methods require arguments to be provided before outputs, avoiding combinational cycles.

Modules need not follow a strict hierarchy, but a hierarchical design has simpler semantics.
Thus, we first focus on modules that follow a strict hierarchy (\autoref{sec:basicmodsemantics}). %and where each module has a single rule.
We then describe the general semantics and additional mechanisms for modules that do not follow a strict hierarchy (\autoref{sec:generalmodsemantics}).

\subsection{Module syntax and elements}
\label{sec:modulesyntax}
\label{sec:submodules}

Modules are defined using the following syntax:
\begin{center}
\verb|module ModType | \optStart \verb|(Type1 arg1, ..., TypeN argN)|\optEnd\verb|;    | \\
\verb|  <submodule, method, input, rule, constant decls> | \\
\verb|endmodule                                          |
\end{center}
Each definition specifies a new module \emph{type}, \verb|ModType|,
which can then be instantiated as a submodule in other modules
or as the top-level module.
Modules meant to be submodules can have optional \emph{module arguments},
\verb|arg|$_{\texttt{i}}$, with types \verb|Type|$_{\texttt{i}}$.
These arguments can be constant values (e.g., used to set initial values),
or other modules.
The body of the module can declare submodules, methods, inputs, rules, and constants,
whose syntax is explained below and shown in the examples to the right.

Modules can be parametric; see \autoref{sec:parametrics} for details.

\paragraph{Submodule declarations} use the syntax:
\begin{center}
  \verb|SubmodType submodName| \optStart \verb|(arg1, ..., argN)|\optEnd\verb|;|
\end{center}
where \verb|SubmodType| is the type of the submodule being instantiated,
\verb|submodName| is the name of this specific submodule instance,
and \verb|arg|$_{\texttt{i}}$ are the (optional) arguments to the module.


\paragraph{Methods} 
are nearly identical to functions:
they specify combinational logic that produces an output
and \emph{have no side effects}.
They can call methods in submodules or read register values,
but they cannot set the inputs of submodules or write to any register
(these are side effects, which only rules can have).

\begin{wrapfigure}{r}{0.32\columnwidth}
\vspace{-2em}
\begin{mscode}
// 8-bit counter that can be
// incremented every cycle
module Counter8;
  Reg#(Bit#(8)) count(0);

  method Bit#(8) getCount;
    return count;
  endmethod

  input Bool increment
    default = False;

  rule tick;
    if (increment)
      count <= count + 1;
  endrule
endmodule

// 16-bit counter built from 
// two Counter8 submodules
module Counter16;
  Counter8 lo(0);
  Counter8 hi(0);

  method Bit#(16) getCount =
   {hi.getCount,lo.getCount};

  input Bool increment
    default = False;

  rule tick;
    if (increment) begin
      lo.increment = True;
      // Increment hi when lo
      // is about to overflow
      hi.increment =
       (lo.getCount == 255);
    end
  endrule
endmodule

\end{mscode}

\vspace{-9em}
\end{wrapfigure}



Methods use a syntax nearly identical to functions:
\begin{center}
\verb|method RetType mname(Type1 arg1, ..., TypeN argN);| \\
\verb|  stmt1                                           | \\
\verb|   ...                                            | \\
\verb|  stmtN                                           | \\
\verb|endmethod                                         |
\end{center}
where \verb|mname| is the method's name;
\verb|RetType| is the type of its return value;
\verb|arg|$_{\texttt{i}}$ are the names of its arguments,
with types \verb|Type|$_{\texttt{i}}$;
and \verb|stmt|$_{\texttt{i}}$ are statements.

Methods also support the same shorthand syntax as functions:
\begin{center}
\verb|method RetType mname(Type1 arg1, ..., TypeN argN) = expr;|
\end{center}

%Methods \emph{may} read module inputs, but this is discouraged
%as it may introduce a combinational cycle.
%See \autoref{sec:advancedmodules} for details.

\subparagraph{Use:}
A method may be called from the methods or rules of its enclosing module.
The syntax for a method call is
\verb|submoduleName.methodName(arg1, ..., argN)|
(\autoref{sec:calls}).


\paragraph{Inputs} specify external inputs that are controlled by
the rule of an enclosing module. Their syntax is:
\begin{center}
\verb|input Type name |\optStart\verb|default = defaultExpr|\optEnd\verb|;|
\end{center}
where \verb|Type| is the input's name,
\verb|name| is its name,
and the optional \verb|defaultExpr| specifies a default value for the input.

\subparagraph{Use:}
Inputs can be read within the module just like variables, but cannot be set.

Inputs can be set within a rule of the enclosing module,
with syntax like a normal assignment, \verb|submoduleName.inputName = expr;|.
If the input does not have a default value,
the enclosing module \emph{must} set the input every cycle.
If the input has a default value, then setting the input is optional.

An input can only be set \emph{once}. Trying to assign to the input multiple
times within a cycle will cause a compiler error.


\paragraph{Rules} specify combinational logic that updates the state of the module,
i.e., they implement side-effects.
Specifically, rules set the values to be written in registers
at the end of the cycle and the inputs of submodules.

Rules use the following syntax:
\begin{center}
  \verb|rule ruleName;| \\
  \verb|  stmt1       | \\
  \verb|   ...        | \\
  \verb|  stmtN       | \\
  \verb|endrule       |
\end{center}

\emph{Rules fire (i.e., automatically execute) every cycle}. A module may have multiple rules,
but these rules cannot have overlapping side-effects
(i.e., they must update disjoint registers and inputs).
Any such overlap will cause a compiler error.

% dsm: Let's keep this separate...
%\paragraph{Differences with BSV:} 

\subsection{Registers}
\label{sec:registers}

Registers are the most basic module.
\verb|Reg#(T)| stores a value of type \verb|T|.
\verb|T| can be any type that can be represented as bits.

\paragraph{Declaration:}
Modules can declare registers with the usual syntax for submodules.
\verb|Reg#(T)| takes an initial value, so its declaration
is \verb|Reg#(T) regName(initialValue);|.

Initial values allow registers to start set to known values.
This requires some additional circuitry (\autoref{sec:sequential}).
If it's not necessary to have an initial value, this circuitry can be avoided by using
\verb|RegU#(T)|, a variant of \verb|Reg#(T)| that starts on an unknown value.
\verb|RegU#(T)| declarations do not take an initial value: \verb|RegU#(T) regName;|

\paragraph{Reads:}
Registers can be read from anywhere in the module.
Simply using the name of the register yields its value.

\paragraph{Writes:}
Registers can be written from rules using the following syntax:
\begin{center}
\verb|regName <= expr;|
\end{center}
where \verb|regName| is the register's name and \verb|expr| is the value to be be written to it.
Note how register writes are not normal assignments: they use \verb|<=| instead of \verb|=|
and have different semantics.

\emph{Register writes do not take place until the end of the cycle.}
Reading a register value in the same rule and after a register write statement
will yield the value of the register in the  \emph{current cycle},
not the value set by the register write.

\emph{Registers can be written only once.}
In each cycle, a register may be written at most once.
A rule that writes the same register multiple times will cause a compiler error.
Two rules that may write to the same register will cause a compiler error.
Registers need not be written every cycle; if not written, a register retains its previous value.

% TODO(dsm): Vector#(Reg#()) ?

\subsection{Semantics of hierarchically nested modules}
\label{sec:basicmodsemantics}

Suppose we impose two conditions on a design.
First, modules follow a strict hierarchy,
i.e., each module interacts only with the submodules it defines.
Second, no method reads module inputs.

Under these conditions, Minispec guarantees that there are no combinational cycles and gives very simple semantics:
the system behaves as if, on each cycle, \emph{rules execute sequentially, outside-in}:
first, the rule in the top-level module fires,
then the rules in all its submodules, and so on.

Because each module's rule calls methods in its submodules and sets the inputs to the submodules,
this order guarantees that all inputs are set by the time a rule executes.
%It also shows that, under these conditions, 
Moreover, the effects of rules in submodules cannot be observed
by their enclosing modules: data flows inside-out only through methods,
and data flows outside-in through inputs and rules.

\subsection{General semantics of modules (advanced)}
\label{sec:generalmodsemantics}

% TODO: Place correctly once earlier content is settled; this inset is not that important
\begin{wrapfigure}{r}{0.32\columnwidth}
\vspace{-4em}
\begin{mscode}
module FIFO;
  input Maybe#(Bit#(4))
   enqueue default = Invalid;
  method Maybe#(Bit#(4))
   first = ...;
  method Bool isFull = ...;
  ...
endmodule

module Prod(FIFO outQ);
  ...
  rule produce;
    if (!outQ.isFull)
     outQ.enqueue = Valid(v);
  endrule
endmodule

module Cons(FIFO inQ);
  ...
  rule consume;
    let f = inQ.first;
    if (isValid(f)) ...
  endrule
endmodule

module TopLevel;
  FIFO queue;
  Prod producer(queue);
  Cons consumer(queue);
  ...
endmodule
\end{mscode}
\vspace{-4em}
\end{wrapfigure}



Though the above semantics are simple, there are some cases for which strict hierarchical nesting can be too restrictive.
For example, suppose we want to connect two modules through a FIFO (First-In-First-Out) queue:
the producer module enqueues values into the queue, and the consumer module dequeues them.
If we followed the hierarchical approach, the top-level module could instantiate three submodules,
\texttt{producer}, \texttt{consumer}, and \texttt{queue}.
Then, the rule in the top-level module could explicitly marshall outputs
from \texttt{producer} into \texttt{queue} and from \texttt{queue} to \texttt{consumer}.
But this can get cumbersome.
Instead, we may want to have \texttt{producer} and \texttt{consumer} both directly
interact with \texttt{queue}, \tmp{as shown in the code to the right}.
But now \texttt{queue}'s methods and inputs are being used by two different modules,
rather than only its enclosing module.
Moreover, there may be non-trivial interactions between them.
For example, if the producer has a value ready but the queue is empty,
we might want to let the consumer dequeue that value on the same cycle.

\paragraph{General semantics:}
In general, rules in arbitrary modules can communicate in a single cycle, as long as
doing so \emph{does not introduce a combinational cycle}.
The compiler will find a fixed order for the rules, and will emit an error on any cycle.

For instance, in our producer-consumer example, the FIFO queue module could have an
\texttt{enqueue} input and a \texttt{first} method that returns that input if the queue has no buffered values.
Because \texttt{consumer} calls \texttt{first}, \texttt{first} reads \texttt{enqueue}, and \texttt{producer}
sets \texttt{enqueue}, \texttt{producer}'s rule is ordered before \texttt{consumer}'s.
If there was another constraint in the system that required the reverse order
(e.g., a queue going in the reverse direction), that would signal a combinational cycle,
which would cause a compiler error.

\paragraph{Wires:}
In general, it may be desirable to split computation across rules and methods
within the same module to avoid combinational cycles and code repetition, or to enforce a particular order.
Wires are modules that allow rules and methods to communicate in the same cycle.
They come in two flavors:
\begin{compactitem}
\item \verb|BypassWire#(T)| implements a wire that must be written on every cycle. It cannot be read until a rule writes to it.
\item \verb|DWire#(T)| implements a wire with a default value. If the wire is not written in a given cycle,
  reads to the wire return the default value. The wire has no memory: the wire ``reverts'' to the default value every cycle.
\end{compactitem}
Wires have the same interface as registers: using the name of the wire yields its value,
assignments with \verb|<=| write to it, only rules may write to wires, and multiple writes are not allowed.

Wires introduce order constraints among rules and methods:
the rule that writes to the wire is ordered before all rules and methods that read the wire.

\subsection{Sequential logic synthesis}
\label{sec:sequential}

Sequential synthesis is a straightforward extension of combinational synthesis (\autoref{sec:combinational}).

Each sequential circuit has inputs and outputs as shown in \autoref{fig:hiermodule}.
Two of these inputs are \emph{implicit} and not present in Minispec code:
CLK, the \emph{clock signal}, and RST, the \emph{reset signal}.

Registers are synthesized as collections of 1-bit D flip-flops (DFFs).
All registers use CLK as their clock.
Registers with an initial value (i.e., \verb|Reg#(T)|) include reset circuitry
that sets its value to the initial value when the circuit powers up.
Because flip-flops hold an arbitrary value when first powered, this is accomplished by
the RST signal: the RST signal is 1 for a few cycles after power-up,
letting registers write their initial values with the circuitry \tmp{shown to the right}.
Registers with no initial value (i.e., \verb|RegU#(T)|) have no reset circuitry.

Registers are always written to by a single rule, but may not be updated every cycle.
When not updated every cycle, the register includes \emph{write-enable} circuitry
to optionally retain its old value, as \emph{shown to the right}.

\todo{Circuit diagrams}

Inputs without a default value are simply wires.
Inputs with a default value translate to a multiplexer that chooses between
the input value, if any is set, and the default value, if none is set, as shown right.

Rules are synthesized as normal combinational logic.
They produce the values for all registers and inputs they set.
When rules conditionally set registers or inputs,
they also generate the corresponding \emph{enable} signals
so that, when not set, registers retain their old value
and inputs use their default value.

Methods are synthesized like functions.
A method can be called multiple times. If the method has no arguments
(i.e., it always returns the same value on a given cycle),
all callers share the same output value. If the method has arguments,
each call instantitates a new copy of the method.

Finally, wires (\autoref{sec:generalmodsemantics})
are synthesized exactly like inputs.

