\section{System Functions}
\label{sec:systemFunctions}

\begin{wrapfigure}{r}{0.32\columnwidth}
\vspace{-3em}
\begin{mscode}
// Test harness for Counter8
// module from Section 9
import counter8;
module Counter8Test;
  Counter8 c;
  Reg#(Bit#(10)) cycle;
  rule tick;
    c.increment =
     (cycle[0] == 1'b1);
    $display("cycle ", cycle,
     ", count ", c.getCount);
    if (cycle == -1) $finish;
    cycle <= cycle + 1;
  endrule
endmodule
\end{mscode}
\vspace{-4em}
\end{wrapfigure}

System functions are useful for simulation and debugging. 
They are used to display information, read and write data,
and terminate the simulation.
They are not synthesizable to hardware,
and are only used when simulating a module.

All system functions begin with a dollar sign (\verb|$|).
System functions are called like normal functions.
System functions \emph{may only be used within module rules}, as they have side effects.
Calls to a system function from a function or method will cause a compiler error.

The two main system functions are \verb|$finish| and \verb|$display|.

\paragraph{\texttt{\$finish}} terminates the simulation. It takes no arguments.

\paragraph{\texttt{\$display}} prints strings to standard output.
It takes a variable number of arguments, which can be strings or other values.

As shown in the examples below, every value that is not a string will be interpreted as an n-bit value and printed as a decimal number by default.
To print numbers in other bases, \verb|$display| can use the same syntax as C's \verb|printf|,
using format strings with \verb|%b| for binary, \verb|%d| for decimal, and \verb|%h| or \verb|%x| for hexadecimal values.

\begin{mscode}
$display("Hello world!");                 // Prints "Hello world!"
$display("Hello", " ", "world!");         // Prints "Hello world!"
// Printing non-string values
Bit#(8) x = 42;
$display("x in decimal is ", x);          // Prints "x in decimal is 42"
// Using printf-style formatting
$display("0b%b == %d == 0h%h", x, x, x);  // Prints "0b00101010 ==  42 == 0x2a"
\end{mscode}

Still, displaying complex types like structs as one long number is inconvenient.
The \textbf{\texttt{fshow}} function automatically formats complex types, as shown in the example below.
\texttt{fshow} can be used on values of any type.
% dsm: i.e., in BSV land, every typedef includes deriving(FShow)

\begin{mscode}
typedef struct { Bit#(8) red; Bit#(8) green; Bit#(8) blue; } Pixel;
Pixel cyan = Pixel{ red : 0, green : 255, blue : 255 };
$display(cyan);         // Prints "   65535"
$display(fshow(cyan));  // Prints "Pixel { red: 'h00, green: 'hff, blue: 'hff }"
\end{mscode}

\texttt{\$display} terminates every printed string with a newline. If you do not want this behavior, use \textbf{\texttt{\$write}},
which uses the same syntax as \texttt{\$display}.

\paragraph{Other system functions:} Minispec can use other system functions from BSV,
e.g., to read standard input or to read and write data from/to files.
BSV system functions are described in Section 12.8 of the
\href{http://csg.csail.mit.edu/6.S078/6_S078_2012_www/resources/reference-guide.pdf}{BSV reference}.
They are rarely needed.
