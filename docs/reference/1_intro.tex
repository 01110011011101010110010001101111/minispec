\section{Introduction}
\label{sec:intro}

This document is the primary reference for the Minispec hardware description language.
It describes the syntax and semantics of the language in detail, and illustrates their use.

\emph{This reference is not intended as an introduction to Minispec}, and assumes familiarity with the language.
We recommend the \href{TODO}{Minispec tutorials} as an introduction to the language.

\paragraph{How to use this reference:}
This reference does not assume you will be reading it sequentially.
Use it to answer specific questions or to improve your knowledge of particular aspects of Minispec.

Each section presents a particular aspect of the language, with sections roughly laid out bottom-up:
the first sections present the basic elements of the language (tokens, types, expression, etc.),
while the latter sections present the more complex elements (functions, modules, etc.), which build on the simpler ones.
Each section presents code examples delimited in \colorbox{codebg}{grey insets}.

% dsm: Eh, whatever, this is pretty standard...
%Each section seeks to be self-contained, so there is some overlap across sections.
%For example, the expressions section (\autoref{sec:expressions}) discusses operators,
%while the built-in types section (\autoref{sec:builtins})
%presents the use of each operator in more detail.
%Cross-references are used to avoid redundancy.

\paragraph{Acknowledgements:}
Minispec is very closely related to Bluespec SystemVerilog (BSV): it shares much of its syntax with BSV
and the Minispec compiler internally translates Minispec to BSV.
Minispec simply would not exist without BSV.
%Nonetheless, Minispec seeks to hide the more advanced aspects of BSV.
Nonetheless, using Minispec requires no knowledge of BSV, and so
this reference is \emph{self-contained}: it presents Minispec on its own, without any further reference to BSV.
\href{TODO}{Minispec from BSV} recounts the differences between both languages.

In writing this reference, we took inspiration from the excellent language references
of \href{TODO}{BSV}, \href{TODO}{Rust}, and \href{TODO}{Python}.

%\paragraph{Notation:}
%In addition to informal descriptions, this reference uses ANTLR's extended Backus-Naur Form (EBNF)
%to precisely specify the Minispec syntax.
%Grammar alternatives are separated by a vertical bar (``\texttt{|}'').
%Items followed by a question mark (``?'') are optional.
%Items followed by an asterisk (``*'') can be repeated zero or more times.
