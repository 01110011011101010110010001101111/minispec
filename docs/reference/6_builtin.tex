\section{Built-In Types}
\label{sec:builtin}

This section describes the five basic built-in types that Minispec provides.
It details the semantics of the operators each type supports and describes
the built-in functions provided to work with these types.

\subsection{Bool}
\label{sec:bool}

\begin{wrapfigure}{r}{0.32\columnwidth}
\vspace{-2.5em}
\begin{mscode}
Bool a = True;
Bool b = False;

Bool x = !a;      // False
Bool y = a && b;  // False
Bool z = a || b;  // True

Bool e = a == b;  // False
Bool n = a != b;  // True
\end{mscode}
\vspace{-2em}
\end{wrapfigure}

Variables of type \verb|Bool| can take one of two values, \verb|True| or \verb|False|.
\verb|Bool| values support the three basic Boolean algebra operations:
Boolean NOT (\verb|!|), Boolean AND (\verb|&&|) and Boolean OR (\texttt{||}).

\verb|Bool| values also support equality operations (\verb|==|, \verb|!=|).
Note that, for Boolean values, \verb|!=| is equivalent to Boolean XOR,
and \verb|==| is equivalent to Boolean XNOR.

\subsection{Bit\#(n)}
\label{sec:bit}

\begin{wrapfigure}{r}{0.32\columnwidth}
\vspace{-2.5em}
\begin{mscode}
Bit#(4) a = 4'b0011;
Bit#(4) b = 4'b0101;

Bit#(4) x = ~a;    // 4'b1100
Bit#(4) y = a & b; // 4'b0001
Bit#(4) z = a ^ b; // 4'b0110

Bit#(4) s = a + b; // 4'b1000
Bool geq = a >= b; // False

Bit#(1) any1 = |a; // 1'b1
Bit#(1) ev1s = ^a; // 1'b0
\end{mscode}
\vspace{-2em}
\end{wrapfigure}

\verb|Bit#(n)| represents an \verb|n|-bit value.
The parameter \verb|n| must be a non-negative \verb|Integer|.

\paragraph{Bitwise logical operations:}
\verb|Bit#(n)| supports bitwise inversion/NOT (\verb|~|), AND (\verb|&|), OR (\texttt{|}), XOR (\verb|^|), and XNOR(\verb|~^| or \verb|^~|).
Bitwise logical operations take \verb|Bit#(n)| inputs and produce a \verb|Bit#(n)| output, where the operations apply to each bit of the inputs.

\paragraph{Arithmetic and relational operations:}
\verb|Bit#(n)| supports all arithmetic and relational operators.
These operations treat \verb|Bit#(n)| values as \emph{unsigned integers}.

\paragraph{Bit reduction operations:}
\verb|Bit#(n)| supports AND (\verb|&|), OR (\texttt{|}), and XOR (\verb|^|) reductions.
Bit reductions are unary operations that take a \verb|Bit#(n)| input
and return a \verb|Bit#(1)| output that results
from reducing the bits of the input using the specified operation.
For example, given \verb|Bit#(4) a|, \verb|&a| is equivalent to \verb|a[3] & a[2] & a[1] & a[0]|.

\paragraph{Converting to and from \texttt{Bit\#(n)}:}
All types except \texttt{Integer} are ultimately represented as a collection of bits in hardware, and can be converted to and from \texttt{Bit\#(n)}.
The built-in function \verb|pack| converts to \verb|Bit#(n)|,
and \verb|unpack| converts from \verb|Bit#(n)|, as shown below.

\begin{mscode}
typedef enum Color {Red, Green, Blue};
Color c = Red;
Bit#(2) x = pack(c);      // 2'b00, the binary representation of Red
Color g = unpack(x + 1);  // Green, whose binary representation is 2'b01
Color u = unpack(x + 3);  // unspecified value, as no Color corresponds to 2'b11
\end{mscode}

\paragraph{Bit selection:}
Given a \verb|Bit#(n)| variable \verb|x|,
\verb|x[i]| selects the \verb|i|$^{th}$ bit of \verb|x|.
\verb|x[i]| has type \verb|Bit#(1)|.

Bits are enumerated \emph{right-to-left, i.e., starting from the least-significant bit}.
For example, given \verb|Bit#(4) x = 4'b1010|,
then \verb|x[0] = 0| (the least-significant or rightmost bit), 
\verb|x[1] = 1|, \verb|x[2] = 0|,
and \verb|x[3] = 1| (the most-significant or leftmost bit).
This is consistent with how digits are always enumerated in a number (according to significance, so the value is $\sum_{i=0}^{n-1}2^ix[i]$),
but is different from how vectors are indexed.

The selector \verb|i| in \verb|x[i]| can be a literal, an \verb|Integer|, or a \verb|Bit#(k)| variable. In the first two cases the bit selected is constant,
which results in an efficient hardware implementation (as no logic gates are required, only wires).
In the last case the bit selected is variable, and will require more hardware to implement.

Given a \verb|Bit#(n)| variable \verb|x|,
\verb|x[i:j]|, with \verb|i| $\geq$ \verb|j|, selects the range of bits of \verb|x|
starting at \verb|x[i]| and ending at \verb|x[j]|, both inclusive.
When the difference between \verb|i| and \verb|j| is known at compile time,
\verb|x[i:j]| has type \verb|Bit#(i-j+1)|.
Otherwise, the bit-width of the selected slice is unknown at compile time,
and on a size mismatch with the destination variable,
the slice will be padded with zeros if the destination is wider,
and the slice will be truncated from the left if the destination is narrower.


\begin{wrapfigure}{r}{0.32\columnwidth}
\vspace{-2em}
\begin{mscode}
Bit#(2) a = 2'b11;
Bit#(4) b = 4'b1001;
Bit#(3) c = 3'b010;
Bit#(9) x = 
  {a, b, c};  // 9'b111001010
let y = {a, c};   // 5'b11010
let z = {1'b1, a};  // 3'b111
\end{mscode}
\vspace{-2.5em}
\end{wrapfigure}

\paragraph{Bit concatenation:}
Given two or more \verb|Bit#()| values $\texttt{x}_\texttt{1},...,\texttt{x}_\texttt{k}$,
$\texttt{\{x}_\texttt{1},...,\texttt{x}_\texttt{k}\texttt{\}}$ concatenates these values. The result has type \verb|Bit#(s)|, where \verb|s| is the sum of the bit-widths of all concatenated elements.

Bit concatenation can take at most one unsized number literal,
and will infer the width of that literal to match the widths
of the concatenated value and the required type, as shown in the examples below.

\begin{mscode}
Bit#(2) a = 2'b11;
Bit#(8) i = {a, 0};            // 8'b1100_0000, as 0 is inferred to be 6 bits
Bit#(8) j = {2'b11, 0, 1'b1};  // 8'b1100_0001, as 0 is inferred to be 5 bits
Bit#(8) k = {~0, ~a, a};       // 8'b1111_0011, as 0 is inferred to be 4 bits
Bit#(8) k = {0, 10};           // Error, can't deduce the sizes of two unsized literals
\end{mscode}

\begin{wrapfigure}{r}{0.32\columnwidth}
\vspace{-2em}
\begin{mscode}
Bit#(4) a = 4'b1001;

Bit#(2) x = truncate(a);
            // 2'b01
Bit#(6) y = zeroExtend(a);
            // 6'b001001
Bit#(6) z = signExtend(a);
            // 6'b111001
Bit#(6) w = signExtend(x);
            // 6'b000001
\end{mscode}
\vspace{-2.5em}
\end{wrapfigure}

\paragraph{Truncation and extension:}
The built-in function \verb|truncate| truncates the most-significant bits of its argument to match the bit-width of a narrower destination. \verb|zeroExtend| adds zeros to the left of the argument to match the bit-width of a wider destination.
Finally, \verb|signExtend| extends the argument by replicating its most significant bit to match the bit-width of a wider destination.

\subsection{Integer}
\label{sec:integer}

\verb|Integer| represents an integer value (i.e., a negative or non-negative number) with an unbounded number of bits.
\verb|Integer| supports the same unary and binary
operators as \verb|Bit#(n)|. However, since \verb|Integer| is signed,
arithmetic and relational operators have signed semantics.

\verb|Integer| is the only type in Minispec that is not synthesizable to hardware, and it is used exclusively at compile time:
all \verb|Integer| expressions must be evaluable by the compiler.
%---intuitively,
%the compiler replaces all \verb|Integer| variables by unsized number literals.
% TODO: The discussion of Integers vs unsized number literals is confucing at this point. BSV had this weird dychotomy, my guess is b/c of numeric types.
% But we can say that unsized number lits -> Integer lits (and sized -> Bit#(n) lits), because that's precisely what's going on!
%As a result, while \verb|Integer| values can be converted to \verb|Bit#(n)| (and this conversion is in fact implicit),
%\verb|Bit#(n)| values, which are not necessarily known statically, cannot be converted to \verb|Integer|.
\autoref{sec:parametrics} explains precisely how \verb|Integer| values are evaluated at compile time.

\subsection{Vector\#(n,T)}
\label{sec:vector}

\begin{wrapfigure}{r}{0.32\columnwidth}
\vspace{-5em}
\begin{mscode}
Vector#(4, Bool) lessThanTwo;
for (Integer i=0; i<4; i=i+1)
  lessThanTwo[i] = i < 2;

// Static selection
Bool a = lessThanTwo[3];
Integer i = 1;
Bool b = lessThanTwo[i+1];
// Error, out-of-bounds
Bool c = lessThanTwo[i+3];

// Dynamic selection
Bit#(3) j = 3'd1;
Bool d = lessThanTwo[j];
// Undefined, out-of-bounds
Bool e = lessThanTwo[j+3];
\end{mscode}
\vspace{-4.5em}
\end{wrapfigure}

% dsm: Deferred
%// Vector of submodules
%module Example;
%  // Vector of 4 8-bit regs,
%  // all initialized to 0
%  Vector#(4, Reg#(Bit#(8)))
%    regVector(4'd0);
%  [...]
%endmodule

\verb|Vector#(n, T)| represents a vector or array of \verb|n| elements of type \verb|T|.

\paragraph{Element selection:} 
Given a \verb|Vector#(n,T)| variable \verb|v|,
\verb|v[i]| selects the \verb|i|$^{th}$ bit of \verb|v|.
\verb|v[i]| has type \verb|T|.

The selector \verb|i| in \verb|v[i]| can be a literal, an \verb|Integer|, or a \verb|Bit#(k)| variable.
As in bit selection (\autoref{sec:bit}), only the last of the three cases results in dynamic selection,
which requires more hardware.

Static and dynamic element selection have different guarantees with respect to out-of-bounds accesses.
Static selectors are always checked by the compiler,
so the compiler will flag any \verb|v[i]| with \verb|i >= n| as an out-of-bounds error.
Dynamic selectors are not checked, so \verb|v[i]| with \verb|i >= n| will return an unspecified element.

\paragraph{Initialization:} Vector variables need not be initialized when declared,
and can be initialized element-by-element.

Modules (\autoref{sec:modules}) can instantiate \emph{Vectors of submodules}.
In this case, the vector is initialized with the same syntax as submodule instantiation (\autoref{sec:submodules}):
\verb|Vector#(n,T) vectorName(submoduleArg1, ..., submoduleArgK);|. Submodule arguments are passed to all submodules.

\paragraph{Other vector functions:} \verb|Vector#(n,T)| is a BSV type.
Appendix C.3 of the \href{TODO}{BSV reference} describes several functions for working with vectors,
which can also be used in Minispec code.

\begin{wrapfigure}{r}{0.32\columnwidth}
\vspace{-4em}
\begin{mscode}
// Creating Maybe#(T)'s
Maybe#(Bit#(1)) a = Valid(1);
Maybe#(Bit#(4)) b = Invalid;

// Validity checking
Bool aValid = isValid(a);

// Unpacking Maybe#(T)'s
let x = fromMaybe(0, a); // 1
let y = fromMaybe(4, b); // 4

// Common unpacking idiom:
// the if condition checks
// validity, so fromMaybe's
// arg is always valid and the
// default value is irrelevant
if (isValid(a)) begin
  let aVal = fromMaybe(?, a);
  [...]
end
\end{mscode}
\vspace{-5em}
\end{wrapfigure}

\subsection{Maybe\#(T)}
\label{sec:maybe}

\verb|Maybe#(T)| represents an optional value of type \verb|T|.
A \verb|Maybe#(T)| can be either \verb|Valid| if it holds a value of type \verb|T|,
or \verb|Invalid| if it does not hold a value.
\verb|Maybe#(T)| is especially useful for modules (\autoref{sec:modules}),
which often do not have valid inputs or outputs every cycle.

\paragraph{Creating \texttt{Maybe\#(T)} values:} Given a value \verb|v| of type \verb|T|,
\verb|Valid(v)| is a valid \verb|Maybe#(T)| that holds \verb|v|.
The literal \verb|Invalid| can be assigned to any \verb|Maybe#(T)|
variable to make it invalid.

\paragraph{Checking for validity:}
The built-in function \verb|isValid| returns \verb|True| if its argument is \verb|Valid|,
and \verb|False| if it is \verb|Invalid|.

\paragraph{Unpacking \texttt{Maybe\#(T)}'s optional value:}
The built-in function \verb|fromMaybe| allows extracting the value of a valid Maybe value.
Its signature is 
\verb|T fromMaybe(T defaultValue, Maybe#(T) x)|.
If \verb|x| is \verb|Valid|, \verb|fromMaybe| returns \verb|x|'s value; if \verb|x| is \verb|Invalid|, \verb|fromMaybe| returns \verb|defaultValue|.


