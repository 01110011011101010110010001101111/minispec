\section{Parametrics and Static Elaboration}
\label{sec:parametrics}

Minispec provides parametric types, such as \texttt{Bit\#(n)},
that take one or more \emph{parameters}, such as \texttt{n} in this case.
These parameters must be known at compile time, and when specified,
yield a concrete type that can be implemented in hardware (such as \texttt{Bit\#(4)}, a 4-bit value).
Minispec also allows defining parametric functions, modules, structs, and type synonyms.

Parametrics enable writing \emph{generic} hardware descriptions
that are then concretized as needed. For example,
we can write a single parametric function \texttt{add\#(n)}
that adds two \texttt{n}-bit numbers.
Then, users of the function will call it with particular values of \texttt{n},
instantiating adder circuits of different widths.
Without parametrics, each such adder would require writing a separate function.

In programming languages, parametrics (also known as generics, templates, or polymorphic types)
primarily improve efficiency and safety: by specializing code at compile time,
parametrics reduce work done at execution time and prevent type errors.
In Minispec, parametrics play a more important role,
because combinational logic has limitations that programming languages do not (it is not Turing-complete).
Thus, \emph{parametrics and compile-time specialization enables functionality that cannot be implemented with combinational logic}.
For example, in a programming language without parametrics (like C),
we could always write a procedure that adds two numbers of arbitrary bitwidths,
e.g., by looping over their bits and adding them one by one.
%(the width would be a runtime \emph{argument} to such procedure).
But we cannot implement a single function (combinational circuit)
that adds numbers of arbitrary widths.

%Parametrics rely on having clear guarantees on what computation is done at compile time.
%Minispec enforces this through the type system:
%all \texttt{Integer} values are computed at compile time.

\paragraph{Parameters} can be either \texttt{Integer}s or \texttt{type}s (including parametric types).

\begin{wrapfigure}{r}{0.32\columnwidth}
\vspace{-2em}
\begin{mscode}
// Doubles a Bit#(n) value
function Bit#(n+1) double
  #(Integer n)(Bit#(n) x)
  = {x, 1'b0};

// n-element FIFO of T's
module FIFO#(Integer n,
             type T);
 Vector#(n, RegU#(T)) elems;
 Reg#(Bit#(log2(n))) head(0);
 Reg#(Bit#(log2(n))) tail(0);
 method Maybe#(T) first;
 ...
endmodule
\end{mscode}
\vspace{-3em}
\end{wrapfigure}




\paragraph{Parametric definitions} follow a consistent syntax.
Parameters are specified right after the function name or module/struct/synonym type,
enclosed in \verb|#(...)|. Specifically:
\begin{compactitem}
\item Functions: \verb|function RetType fname#(PDef1, ..., PDefN) ...|.
\item Modules: \verb|module ModType#(PDef1, ..., PDefN) ...|.
\item Structs: \verb|typedef struct { ... } StructType#(PDef1, ..., PDefN);|.
\item Synonyms: \verb|typedef Type NewType#(PDef1, ..., PDefN);|.
\end{compactitem}
and \verb|PDef|$_{\texttt{i}}$ are the \emph{parameter definitions}, each of which can be:
\begin{compactitem}
\item \texttt{Integer intParam} to denote an \texttt{Integer} parameter with name \texttt{intParam}, which must be lowercase.
\item \texttt{type TypeParam} to denote a \texttt{type} parameter with name \texttt{TypeParam}, which must be uppercase.
\item A specific Integer literal or type, which allows writing parametric definitions that are already specialized to particular parameter values
  (see discussion on \emph{specialization} below).
\end{compactitem}
Integer and type parameter names can be used anywhere in the parametric definition,
including in the return type of the function.

%\begin{wrapfigure}{r}{0.32\columnwidth}
%\vspace{-2em}

%\begin{minipage}{0.49\columnwidth}
\begin{listing}
\begin{minipage}[b]{0.49\columnwidth}
  \begin{mscode}
// n-bit ripple-carry adder with carry-in
// using a single parametric function
function Bit#(n+1) rca#(Integer n)
  (Bit#(n) a, Bit#(n) b, Bit#(1) cin);
 if (n == 1) begin
  let cout = (a & b) | (a & cin) | (b & cin);
  let sum = a ^ b ^ cin;
  return {cout, sum};
 end else begin
  let x = rca#(n-1)(a[n-2:0], b[n-2:0], cin);
  let u = rca#(1)(a[n-1], b[n-1], x[n-1]);
  return {u, x[n-2:0]};
 end
endfunction
\end{mscode}
\end{minipage}
\hfill
\begin{minipage}[b]{0.49\columnwidth}
\begin{mscode}
// n-bit ripple-carry adder with carry-in
// base case (specialized)
function Bit#(2) rca#(1)
  (Bit#(1) a, Bit#(1) b, Bit#(1) cin);
 let cout = (a & b) | (a & cin) | (b & cin);
 let sum = a ^ b ^ cin;
 return {cout, sum};
endfunction

// general case (used when n != 1)
function Bit#(n+1) rca#(Integer n)
  (Bit#(n) a, Bit#(n) b, Bit#(1) cin);
 let x = rca#(n-1)(a[n-2:0], b[n-2:0], cin);
 let u = rca#(1)(a[n-1], b[n-1], x[n-1]);
 return {u, x[n-2:0]};
endfunction
\end{mscode}
\end{minipage}

\vspace{-0.22in}
\hbox{\hspace{3.23in} {(a)} \hspace{3.33in} {(b)}}
\caption{Two ripple-carry adder implementations using recusive parametric functions. (a) uses a single function,
whereas (b) splits the base case into a separate specialized parametric function definition.}
\label{lst:rcas}
\end{listing}

%\vspace{-3em}
%\end{wrapfigure}

\emph{Parametric definitions can be recursive on parameters}.
For example, function \texttt{rca\#(n)} in \trailref{lst:rcas}{(a)} calls \texttt{rca\#(n-1)}.
This recursion is unrolled at compile time, so it translates to a chain of function instances.

\emph{Parametric definitions can be specialized}.
The code can specify both a general parametric definition with Integer and type parameters,
and one or more specialized definitions with fixed integers and types.
On a match, a specialized definition takes precedence over the general one.
For example, an alternative defintion of function \texttt{rca\#(n)}, shown in \trailref{lst:rcas}{(b)},
includes a specialized definition of \texttt{rca\#(1)} as a base case to stop recursion.

\paragraph{Static elaboration:} Parametrics require clear guarantees on what computations
and simplifications the compiler performs at compile time.
We refer to this process as static elaboration.

Guarantees are important because we may want to use parametrics with expressions.
For example, \trailref{lst:rcas}{(a)} calls \texttt{rca\#(n-1)},
and we must guarantee that all parameters are computed at compile time.

Minispec guarantees the following static elaboration behavior:
\begin{compactenum}
\item All \texttt{Integer} expressions and variables are elaborated, i.e., turned into concrete values at compile time.
  An \texttt{Integer} expression or variable that cannot be elaborated causes a compiler error.
\item \texttt{Bool} expressions and variables are elaborated if they depend only on \texttt{Bool} constants and \texttt{Integer} values.
\item All conditional expressions (ternary, case) and statements (if-else, case)
  whose predicate is an elaborated \texttt{Bool} or \texttt{Integer} expression are
  elaborated to include only the branch that they execute, eliminating all others.
\item Parametrics are instantiated and elaborated lazily, as the compiler finds code that uses them.
\end{compactenum}

To show how these rules combine, consider the example in
\trailref{lst:rcas}{(a)}.
Assume that some other code calls \verb|rca#(3)|.
This triggers the instantiation of \verb|rca#(Integer n)| with \verb|n=3|.
Since \verb|n == 1| is elaborated to \verb|False|, the \verb|if| branch is eliminated and all that remains is
the \verb|else| branch.
This code calls \verb|rca#(2)| (by elaborating \verb|rca#(n-1)|) and \verb|rca#(1)|.
The compiler then instantiates \verb|rca#(2)| (and leaves \verb|rca#(1)| pending).
\verb|rca#(2)| is elaborated just like \verb|rca#(3)|: the \verb|if-else| is turned into the \verb|else| branch,
which calls \verb|rca#(1)| twice.
Finally, the compiler instantiates \verb|rca#(1)|.
\verb|n == 1| is elaborated to \verb|True|, so the \verb|else| branch is eliminated and all that remains is the
\verb|if| branch. Since this branch does not call any functions,
no more parametrics are needed and static elaboration terminates.

From the example above, note that recursion stops by relying on elimination of the \verb|else| branch in \verb|rca#(1)|.
If this did not happen (i.e., if the \verb|if|'s predicate could not be elaborated),
then the \verb|else| branch would trigger a call to \verb|rca#(0)| and recursion would not end.
(To avoid infinite recursion, the compiler limits recursion depth and emits an error when exceeded.)

