\section{Expressions}
\label{sec:expressions}

\emph{Expressions} are a combination of variables, literals, operators, and function calls that are \emph{evaluated} to produce a value.
Expressions appear in the right-hand side of assignments, as arguments to function calls, and generally anywhere a value is needed.

This section describes the different types of expressions and their syntax, but \emph{does not detail the semantics of most expression types}.
For example, the operators subsection below explains the syntax and precedence of operators, but does not detail what each one \emph{does}.
This is because these semantics are type-specific, so they are better explained later (e.g., in \autoref{sec:builtin} for operators).

\paragraph{Simple and complex expressions:}
Expressions may be as simple as a single variable or literal, may involve a single operator or function call (e.g., \texttt{a \&\& b} or \texttt{foo(x)}),
or may consist of multiple nested \emph{subexpressions} (e.g., \texttt{foo(a \&\& b) || c}).

\paragraph{Parentheses, precedence, and associativity:}
Complex expressions with multiple operators can use parentheses to explicitly specify the evaluation order of the multiple operations
(e.g., \texttt{(a + b) * c} or \texttt{a + (b * c)}).

When parentheses are not used, evaluation order is determined using the \emph{precedence} and \emph{associativity} rules of the different operators.
Each operator has a precedence, and higher-precedence operators are evaluated first.
For example, \texttt{a + b * c} is equivalent to \texttt{a + (b * c)} because \texttt{*} has higher precedence than \texttt{+}.

Among operators with the same precedence, associativity determines evaluation order.
Almost all operators are \emph{left-associative}, so the expression is evaluated left-to-right.
For example, \verb|x << 4 << 2| is equivalent to \verb|(x << 4) << 2| (i.e., \verb|x << 6|),
and \emph{not} \verb|x << (4 << 2)| (which would be \verb|x << 16|).
The only right-associative operator in Minispec is the ternary operator (\autoref{sec:expressions:ternary}).

Though these rules match those of most programming languages and should feel natural,
we recommend using parentheses when the evaluation order is not immediately obvious.

\paragraph{No side effects:} Expressions in Minispec \emph{never} have side effects,
i.e., evaluating an expression never modifies any variable or register.
This means that independent subexpressions in a complex expression can be evaluated in any order (or in parallel).
For example, in \texttt{foo(a \&\& b) || foo(b)}, both calls to \texttt{foo}
can be evaluated in any order (function calls have no side effects).

% TODO: Tool box
\paragraph{Evaluating standalone expressions:} The Minispec tools let you type and evaluate standalone expressions.
Use \texttt{ms eval} from the command line (e.g., \verb|ms eval File.ms "a + b"|)
or the \texttt{\%\%eval} magic from Jupyter (e.g., \texttt{\%\%eval a + b}).

% dsm: No, defer to static elab, this is much more about semantics
%\paragraph{Compile-time vs.\ run-time evaluation:}

\subsection{Unary and binary operators}

% dsm: Eh, I think this is going to be way too redundant with the more natural type-centric descriptions and examples in the built-in types section. Skip.
\begin{comment}
\begin{tabular}{cc|c}
\multicolumn{2}{c|}{\textbf{Boolean operators}} & \textbf{Examples} \\
%\multicolumn{2}{c}{Apply to \verb|Bool|} & x\verb|Bool x = True; Bool y = False;| \\
\multicolumn{2}{c|}{\multirow{2}{*}{Apply to \texttt{Bool}}} & \verb|Bool x = True;| \\
\multicolumn{2}{c|}{}                                        & \verb| Bool y = False;| \\
\hline
\verb|!| & Boolean NOT  & \verb|    !x = False| \\
\verb|&&| & Boolean AND & \verb|x && y = False| \\
\texttt{||} & Boolean OR & \texttt{x || y = True } \\
\end{tabular}

\begin{tabular}{cc|c}
\multicolumn{2}{c|}{\textbf{Equality operators}} & \textbf{Examples} \\
\multicolumn{2}{c|}{\multirow{2}{*}{Apply to all types}} & \verb|Bit#(2) x = 2'b11;| \\
\multicolumn{2}{c|}{}                                    & \verb|Bit#(2) y = 2'b10;| \\
\hline
\verb|==| & Equal     & \verb|x == y = False| \\
\verb|!=| & Not equal & \verb|x != y = True | \\
\end{tabular}

\begin{tabular}{cc|c}
\multicolumn{2}{c|}{\textbf{Comparison operators}} & \textbf{Examples} \\
\multicolumn{2}{c|}{Apply to \texttt{Bit\#(n)},} & \verb|Bit#(2) x = 2'b11;| \\
\multicolumn{2}{c|}{ \texttt{Integer}}           & \verb|Bit#(2) y = 2'b10;| \\
\hline
\verb|>|  & Greater          & \verb|x == y = False| \\
\verb|>=| & Greater-or-equal & \verb|x != y = True | \\
\end{tabular}
\end{comment}

% dsm: Initial version of operators; too long!
\begin{comment}
\begin{minipage}[t]{0.5\columnwidth}
\centering
\begin{tabular}{cccc}
  \toprule
  \multicolumn{4}{c}{\textbf{Unary operators}} \\
  \textbf{Type}    & \textbf{Applies to} & \textbf{Operators} & \textbf{Names} \\
  \midrule
  \textbf{Boolean} & \verb|Bool| & \verb|!| & Boolean NOT \\
  \midrule
  \textbf{Logical} & \verb|Bit#(n)|, \verb|Integer| & \verb|!| & Bitwise invert \\
  \midrule
  \textbf{Arithmetic} & \verb|Bit#(n)|, \verb|Integer| & \verb|~| &  \\
  \midrule
  \textbf{Reduction} & \verb|Bit#(n)|, \verb|Integer| & \verb|&|, \textbf{|}, \verb|^| & AND/OR/XOR bit reduction \\
  \bottomrule
\end{tabular}
\end{minipage}
\hfill
\begin{minipage}[b]{0.5\columnwidth}
  \centering
\begin{tabular}{cccc}
  \toprule
  \multicolumn{4}{c}{\textbf{Binary operators in precedence order}} \\
  \textbf{Type}    & \textbf{Applies to} & \textbf{Operators} & \textbf{Names} \\
  \midrule
  \multirow{4}{*}{\textbf{Arithmetic}} & \verb|Bit#(n)|, \verb|Integer| & \verb|**| & Exponentiation \\
  & & \verb|*|, \verb|/|, \verb|%| & Multiplication, division, modulus \\
  & & \verb|+|, \verb|-| & Addition, subtraction \\
  & & \verb|<<|, \verb|>>| & Left shift, right shift \\
  \textbf{Relational} & \verb|Bit#(n)|, \verb|Integer| & \verb|<|, \verb|>|,\verb|<=|, \verb|>=| & Less/greater-than, less/greater-or-equal \\
  \midrule
  \textbf{Equality} & all types &  \verb|==|, \verb|!=| & Equality, inequality \\
  \midrule
  \multirow{4}{*}{\textbf{Logical}} && \verb|&| & Bitwise AND \\
  && \verb|^| & Bitwise XOR \\
  && \verb|^~|, \verb|~^| & Bitwise XNOR \\
  && \texttt{|} & Bitwise OR \\
  \midrule
  \multirow{2}{*}{\textbf{Boolean}} && \verb|&&| & Boolean AND \\
  && \texttt{||} & Boolean OR \\
  \bottomrule
\end{tabular}
\end{minipage}
\end{comment}

\begin{minipage}[t]{0.45\columnwidth}

\vspace{-1.4in}
The tables below and to the right detail Minispec's unary and binary operators, respectively.
Unary operators always have higher precedence than binary operators
(so, for example, \verb|!a && b| is equivalent to \verb|(!a) && b|).
Binary operators are shown in descending precedence order.
Operators in the same row have the same precedence.
All binary operators are left-associative.
\\

\vspace{8.3pt}

\centering
\begin{tabular}{ccc}
  \toprule
  \multicolumn{3}{c}{\textbf{Unary operators}} \\
  \textbf{Type}    & \textbf{Operators} & \textbf{Names} \\
  \midrule
  \textbf{Boolean} & \verb|!| & Boolean NOT \\
  \midrule
  \textbf{Logical} & \verb|~| & Bitwise invert \\
  \midrule
  \textbf{Arithmetic} & \verb|+|, \verb|-| & Unary plus, negation \\
  \midrule
  \textbf{Reduction} & \verb|&|, \textbf{|}, \verb|^| & AND/OR/XOR bit reduction \\
  \bottomrule
\end{tabular}
\end{minipage}
\hfill
\begin{minipage}[t]{0.5\columnwidth}
  \centering
\begin{tabular}{ccc}
  \toprule
  \multicolumn{3}{c}{\textbf{Binary operators in precedence order}} \\
  \textbf{Type}    & \textbf{Operators} & \textbf{Names} \\
  \midrule
  \multirow{4}{*}{\textbf{Arithmetic}} & \verb|**| & Exponentiation \\
  & \verb|*|, \verb|/|, \verb|%| & Multiplication, division, modulus \\
  & \verb|+|, \verb|-| & Addition, subtraction \\
  & \verb|<<|, \verb|>>| & Left shift, right shift \\
  \midrule
  \textbf{Relational} & \verb|<|, \verb|>|,\verb|<=|, \verb|>=| & Less/greater-than/or-equal \\
  \midrule
  \textbf{Equality} & \verb|==|, \verb|!=| & Equality, inequality \\
  \midrule
  \multirow{4}{*}{\textbf{Logical}} & \verb|&| & Bitwise AND \\
  & \verb|^| & Bitwise XOR \\
  & \verb|^~|, \verb|~^| & Bitwise XNOR \\
  & \texttt{|} & Bitwise OR \\
  \midrule
  \multirow{2}{*}{\textbf{Boolean}} & \verb|&&| & Boolean AND \\
  & \texttt{||} & Boolean OR \\
  \bottomrule
\end{tabular}
\end{minipage}

\smallskip

Most operators admit operands of particular types. Specifically:
\begin{compactitem}
\item Boolean operators apply only to values of type \texttt{Bool}.
\item Arithmetic, relational, bitwise logical, and bit reduction operators
  apply only to values of types \texttt{Bit\#(n)} or \texttt{Integer}.
\item Equality operators apply to values of any type, including user-defined types.
\end{compactitem}

\subsection{Conditional expressions}
\label{sec:condExprs}

Conditional expressions allow selecting between two or more values depending on another value.

\paragraph{Conditional operator:} The conditional or ternary operator selects between two values based on a Boolean value.
Its syntax is: 
\begin{center}
\verb|condExpr? trueExpr : falseExpr|
\end{center}
where \verb|cond| is a \verb|Bool| expression, and \verb|trueExpr| and \verb|falseExpr| are expressions of the same type.
If \verb|cond| is \verb|True|, the expression evaluates to \verb|trueExpr|; otherwise, it evaluates to \verb|falseExpr|.

\emph{The conditional operator is right-associative} (in fact, it is the only right-associative operator).
Therefore, \verb|a? x : b? y : z| is equivalent to
\verb|a? x : (b? y : z)|, and \emph{not} \verb|(a? x : b)? y : z|.

The conditional operator has the lowest precedence of all other operators.
Therefore, \verb|a? x : y + z| is equivalent to \verb|a? x : (y + z)|,
and \emph{not} \verb|(a? x : y) + z|.

\begin{wrapfigure}{r}{0.32\columnwidth}
\vspace{-2em}
\begin{mscode}
Bit#(2) v = 2'b10;

Bool isOdd = case (v)
  2'b01 : True;
  2'b11 : True;
  default : False;
endcase

// Can omit default if we
// list all possibilities
Bit#(2) plusOne = case (v)
  0 : 1;
  1 : 2;
  2 : 3;
  3 : 0;
endcase
\end{mscode}
\vspace{-2em}
\end{wrapfigure}

\paragraph{Case expression:} The case expression allows selecting among multiple values depending on a value. Its syntax is:
\begin{center}
\verb|case (compExpr)           | \\
\verb|  value1 : expr1;         | \\
\verb|  value2 : expr2;         | \\
\verb|        ...               | \\
\verb|  | \tmp{$[$} \verb|default : defaultExpr;  |\tmp{$]$} \\
\verb|endcase                   |
\end{center}
where \verb|compExpr|, \verb|expr|$_{\verb|i|}$, and \verb|defaultExpr| are expressions,
and \verb|value|$_{\verb|i|}$ are different values of the same type as \verb|compExpr|.
If \verb|compExpr|'s value matches one of the values \verb|value|$_{\verb|i|}$,
the case expression evaluates to its corresponding expression \verb|expr|$_{\verb|i|}$.
Otherwise, the case expression evaluates to \verb|defaultExpr|.

The case expression must match against something---it must always evaluate to a value.
Therefore, the \verb|default| item is optional only when the case expression enumerates
all possible values of \verb|compExpr|.
A case expression that does not enumerate all possible values and does not have a default
item will produce a compiler error.

%\todo{IIRC there was some Bluespec oddity regarding parentheses on case expressions, maybe due to pattern matching. If that's the case, GO FIX IT IN THE COMPILER.}
%
%\todo{Remove blockExpr from Minispec syntax, turn blockExpr into blockStmt; ditto for returnExpr, make it a Stmt; and nix the ``expression ;'' statement.}
%
%\tmp{Subsections below still TODO, are the same as Bluespec. Might want to merge all type-specific expressions under the same heading. Also, add examples to case statements.}

\subsection{Selection and concatenation expressions}

Selection expressions use brackets (\verb|[]|) to select a bit or range of bits from a \verb|Bit#(n)| value,
or an element from a \verb|Vector#(n,T)|.
Their syntax is \verb|expr[ startExpr|\tmp{$[$}\verb|:endExpr|\tmp{$]$}\verb|]|.
\autoref{sec:bit} describes the semantics of selection expressions for \verb|Bit#(n)|,
and \autoref{sec:vector} describes the semantics for \verb|Vector#(n,T)|.

Concatentaion expressions use curly braces (\verb|{}|) to concatenate multiple \verb|Bit#(n)| values into a wider value.
Their syntax is \verb|{expr1, ..., exprN}|.
\autoref{sec:bit} describes their semantics.

\subsection{Struct creation expressions}
\label{sec:structExpr}

%Structs are described in detail in \autoref{sec:structs}.
The struct creation expression creates a struct value. Its syntax is
%\begin{center}
%\verb|StructType{member1 : expr1, ..., memberN : exprN}|.
%\end{center}
\texttt{StructType\{member1\,:\,expr1, ..., memberN\,:\,exprN\}}.
%where \verb|StructType| is the struct's type name, \verb|member|$_{\verb|i|}$ are the struct members' names,
%and \verb|expr|$_{\verb|i|}$ denote the values that the members should take.
%See \autoref{sec:structs} for semantics and examples.
\autoref{sec:structs} describes structs and the semantics of struct creation.

\subsection{Function and method calls}
\label{sec:calls}

Functions are described \autoref{sec:functions}. The function call expression allows invoking a function. Its syntax is:
\begin{center}
  \verb|funcName|\tmp{$[$}\verb|#(param1, ..., paramK)|\tmp{$]$}\tmp{$[$}\verb|(argExpr1, ..., argExprN)|\tmp{$]$}
\end{center}
where \verb|funcName| is the name of the function, \verb|param|$_{\verb|i|}$ are the function's parameters,
and \verb|argExpr|$_{\verb|i|}$ are the function's arguments.
Parameters must be specified if the function is parametric (\autoref{sec:parametrics}), and can't be specified otherwise.
Eanch parameter must be a type or an \verb|Integer| expression.
Arguments are optional because it's possible to have functions without arguments. 

Methods allow modules to return values, and are described in \autoref{sec:methods}. The method call expression has the following syntax:
\begin{center}
    \verb|submoduleName.methodName|\tmp{$[$}\verb|(argExpr1, ..., argExprN)|\tmp{$]$}
\end{center}
where \verb|submoduleName| is the name of the submodule instance that implements the method,
\verb|methodName| is the name of the method, and \verb|argExpr|$_{\verb|i|}$ are the method's arguments.
Methods cannot be parametric.
