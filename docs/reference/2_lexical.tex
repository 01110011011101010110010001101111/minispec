\section{Lexical Structure}
\label{sec:lexical}

This section describes the types of tokens or lexemes in Minispec, i.e., its most basic syntax elements.

\subsection{Whitespace and comments}

Spaces, tabs, and newlines all constitute whitespace.
Minispec is a \emph{free-form language}: whitespace serves only to separate tokens and has no other significance.

\begin{wrapfigure}{r}{0.32\columnwidth}
\vspace{-2em}
\begin{mscode}
// Example one-line comment
/* You can write comments
   across multiple lines */
\end{mscode}
\vspace{-2em}
\end{wrapfigure}
Comments are treated as whitespace, and can either be one-line comments starting with \texttt{//},
or multiline comments delimited by \texttt{/*} and \texttt{*/}. Comments do not nest.

\subsection{Identifiers and capitalization}
\label{sec:identifiers}

\begin{wrapfigure}{r}{0.32\columnwidth}
\vspace{-2em}
\begin{mscode}
// Uppercase identifiers
Bool
FooType

// Lowercase identifiers
fooVar
barFunction

// Dollar-sign identifiers
$display
$finish
\end{mscode}
\vspace{-4em}
\end{wrapfigure}

Identifiers, also called names, are non-empty strings where:
\begin{compactitem}
\item The first character is a lowercase letter, an uppercase letter, or a dollar sign (\texttt{\$}).
\item The remaining characters are alphanumeric or underscore (\texttt{\_}).
\end{compactitem}

Capitalization is important in Minispec.
Identifiers for type names, module names, and enum labels \emph{must begin with an uppercase letter}.
Every other identifier (i.e., those for variables, function names, method names, rule names, input names, and instance names)
\emph{must begin with a lowercase letter}.

Identifiers whose first character is \texttt{\$} are reserved for \emph{system functions}
(see \autoref{sec:systemFunctions}).

% dsm: Fit keywords in page 1
\vspace{-0.02in}
\subsection{Keywords}

The following words are reserved \emph{Minispec keywords}, and cannot be used as normal identifiers:
\vspace{-0.06in}

% dsm: Import/bsvimport @ end now to avoid messing up with syntax
\begin{mscode}
type       function       method       input      begin    for          let
typedef    endfunction    endmethod    default    end      return
struct     module         rule         case       if       import
enum       endmodule      endrule      endcase    else     bsvimport    
\end{mscode}

In addition, because Minispec compiles to BSV and Verilog, the Minispec compiler currently forbids using keywords from these languages as identifiers.
The compiler will produce an error if any of these keywords is used.

\subsection{Number literals}

\begin{wrapfigure}{r}{0.32\columnwidth}
\vspace{-2em}
\begin{mscode}
// Sized number literals
// 4-bit decimal value 10
4'd10    // in decimal
4'b1010  // in binary
4'ha     // in hex
4'hA     // hex digits are
         // case-insensitive
// 8-bit decimal value 10
8'b00001010
8'b1010
8'h0a

// Unsized number literals
'd10  // decimal 10
10    // also decimal 10
'b1010
'habcdef

// _ can separate digits
16'b1010_0110_1101_0010
\end{mscode}
\vspace{-2em}
\end{wrapfigure}

Number literals encode numeric values. They can be \emph{sized} or \emph{unsized}.
Sized literals encode their bit-width, i.e., the number of bits they take.
By contrast, unsized literals have no explicit bit-width.
Literals can be specified in decimal, hexadecimal, and binary bases.

A sized number literal always consists of three elements:
\begin{compactenum}
\item Bit-width, written as a decimal number.
\item Base: \texttt{'d} for decimal, \texttt{'h} for hexadecimal, and \texttt{'b} for binary.
\item Value, written using digits in the specified base (0--9 for decimal, 0--9 and a--f for hex, and 0 or 1 for binary).
\end{compactenum}

An unsized number literal has no bit-width, and consists of an optional base and value.
If the base is not given, the value is interpreted as decimal.

To make long values more readable, number literals allow
using underscore characters (\_) to separate value digits.

Sized and unsized literals can both be used in expressions involving \verb|Bit#(n)| variables.
The examples below illustrate their pros and cons.
Sized literals enforce their bit-width, and will cause a compile-time error on a width mismatch.
By contrast, unsized literals have their bit-width deduced by the compiler.
This makes code more succint and can help express long values,
but the compiler won't catch mistakes that stem from wrong assumptions about bit-widths.

\begin{mscode}
Bit#(4) x = 4'd10;  // OK, as both x and the literal are 4 bits
Bit#(5) y = 4'd10;  // Error due to mismatched bit-widths, y is 5 bits
Bit#(4) z = 10;     // OK, 10 inferred to be 4 bits
Bit#(4) w = 1000;   // Error, decimal 1000 inferred to be 4 bits but doesn't fit in 4 bits
\end{mscode}

Unsized literals must be used in expressions involving \verb|Integer| variables,
as \verb|Integer| is an unsized, compile-time-only type (see \autoref{sec:integer}).

\subsection{String literals}

\begin{wrapfigure}{r}{0.32\columnwidth}
\vspace{-2em}
\begin{mscode}
"This is a string"
"String with a\nline break"
\end{mscode}
\vspace{-2em}
\end{wrapfigure}

String literals are enclosed in double quotes (\verb|"..."|) and must be written on a single line.
String literals admit special characters using the following escape sequences:

\begin{tabular}{rrrrr}
  \emph{Character} & newline & tab & backslash & double quote \\
  \emph{Escape sequence} & \verb|\n| & \verb|\t| & \verb|\\|& \verb|\"| \\
\end{tabular}

\subsection{Bool and Maybe literals}

\verb|True| and \verb|False| are literals of type \verb|Bool| (\autoref{sec:bool}).
\verb|Invalid| and \verb|Valid| are literals of type \verb|Maybe| (\autoref{sec:maybe}).
Like keywords, these literals cannot be used as normal identifiers.

\subsection{Don't-care values}

A question mark (?) denotes a special don't-care value.
Don't-care values can be assigned to variables of any type and can be used in place of literals.
They help the compiler produce better circuits (see \autoref{sec:combinational} for examples). %, as shown in the examples.
%TODO: Examples

