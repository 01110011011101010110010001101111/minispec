\section{Functions and Combinational Logic}
\label{sec:functions}

Functions implement combinational logic (also known as Boolean logic).
Each function takes one or more \emph{input arguments} and produces an
\emph{output result} that depends only on the values of the arguments,
and does not depend on any other state.

\begin{wrapfigure}{r}{0.32\columnwidth}
\vspace{-1em}
\begin{mscode}
function Bool and2(Bool a,
                   Bool b);
  return a && b;
endfunction

function Bool and3(Bool a,
                   Bool b,
                   Bool c);
  Bool ab = and2(a, b);
  return and2(ab, c);
endfunction

// Functions can have 
// multiple return statements
function Bool maj(Bit#(3) x);
  Bit#(2) sum = {0, x[0]} +
                {0, x[1]} +
                {0, x[2]};
  if (sum >= 2) return True;
  else return False;
endfunction

// Shorthand syntax
function Bit#(1) p(Bit#(3) x)
  = x[0] ^ x[1] ^ x[2];
\end{mscode}
\vspace{-6em}
\end{wrapfigure}

%// Functions are first-class
%// objects
%function Bool c(Bool a,
%                Bit#(3) x);
%  let fn = a ? maj : par;
%  return fn(a);
%endfunction

\paragraph{Definition:}
Functions can be defined using the following syntax:
\begin{center}
\verb|function RetType fname(Type1 arg1, ..., TypeN argN);| \\
\verb|  stmt1                                             | \\
\verb|   ...                                              | \\
\verb|  stmtN                                             | \\
\verb|endfunction                                         |
\end{center}
where \verb|fname| is the function's name;
\verb|RetType| is the type of its return value;
\verb|arg|$_{\texttt{i}}$ are the names of its input arguments,
with types \verb|Type|$_{\texttt{i}}$;
and \verb|stmt|$_{\texttt{i}}$ are statements.

Functions must return a result by using \textbf{return statements}. %(\autoref{sec:retStmt}). % dsm: No backref, this is now the primary def for return statement even though sec:statements has a fwd pointer.
As shown in the examples, simple functions without any conditional statements
should end with a single return statement.
Functions can use multiple return statements, e.g., in different branches
of conditional (if/else or case) statements.
A function that does not always return a value will produce a compiler error.

Functions can return only one value. To emulate returning multiple values, the
function can return a struct (\autoref{sec:structs}) instead.
%
A function may have no arguments.

Functions can be parametric; see \autoref{sec:parametrics} for details.

\paragraph{Definition by assignment:}
Short functions consisting of a single expression can be declared more succinctly using the following syntax:
\begin{center}
\verb|function RetType fname(Type1 arg1, ..., TypeN argN) = expr;|
\end{center}
where \verb|expr| is an expression of type \verb|RetType| that computes the return value,
and all other elements are as above.

% dsm: Skipping 1st-class object stuff, because:
% (1) it's needlessly advanced for 6.004 
% (2) we don't have proper Minispec Function types yet that obey Minispec syntax (BSV function types are "function ...", but that breaks our strict "all types are uppercase" rule, so we should have some Function#(RetType,ArgType1...,ArgTypeN) variadic parametric type, perhaps with compiler support since I'm not sure BSV has variadics (maybe with typeclasses?).
% (3) functions that take/return functions... cool, but makes synthesis explanation harder.
%\paragraph{Use:} Call expressions (\autoref{sec:calls}) allow invoking a function.
%Moreover, \emph{functions are first-class objects}, and can be assigned to variables or used anywhere a variable is used.

\paragraph{Restrictions:}
Functions must obey two restrictions to be synthesizable to hardware:
\begin{compactenum}
\item No \texttt{Integer} arguments or an \texttt{Integer} result
  (\texttt{Integer}s may only be parameters, see \autoref{sec:parametrics}).
\item No recursion
  (parametric functions can perform recursion on \texttt{Integer} parameters
  because all such recursive calls are unfolded at compile time, see \autoref{sec:parametrics}).
\end{compactenum}

\subsection{Combinational logic synthesis}
\label{sec:combinational}

Functions are always synthesizable into combinational circuits, i.e.,
circuits with digital inputs and outputs, where each output is a Boolean function of the inputs,
and where each output is guaranteed to reach a stable digital value after
a bounded \emph{propagation delay} from the time at which the inputs reach valid digital values.
Combinational circuits are a basic building block of digital logic.
Here, we discuss \emph{why} Minispec functions are always synthesizable into combinational circuits, and
sketch \emph{how} this synthesis is done in practice.

\paragraph{Function properties:}
Minispec functions have three properties that differ from
functions in other programming languages and that enable synthesis to combinational logic:
\begin{compactenum}
\item \emph{Pure}: Functions compute the output based \emph{only} on the values of its input arguments.
  They cannot use or alter any state or variables outside of the function
  (they may use global constants, however).
  Thus, given a particular set of imput arguments, a function always produces the same output.
\item \emph{Acyclic}: Functions have no cycles, i.e., they cannot ``jump back'' to a prior point of execution. 
  % FIXME(dsm): Needs better explanation.
  They are thus always guaranteed to terminate.
  This is because \emph{(i)} recursion is not allowed, and \emph{(ii)} for loops have known and fixed iteration counts.
\item \emph{Synthesizable arguments and output}: Each input and output of a function is always
  of a type that is representable using a fixed number of bits.
\end{compactenum}

\paragraph{Why functions are synthesizable:}
Given the above properties, it is easy to show that all functions can be synthesized to combinational circuits.
Since each input and output takes a fixed number of bits, we can always
enumerate all input values, build a truth table (computing the output for each input),
and implement the truth table as a combinational circuit (e.g., as a sum-of-products).
While this is a simple proof, it is not \emph{how} circuits are actually synthesized,
as building a truth table and deriving an optimized implementation from it
takes too much space and time for circuits with more than a few bits of inputs.

\paragraph{How functions are synthesized:}
In practice, Minispec functions are synthesized by composing smaller building blocks.
This exploits that combinationsl circuits \emph{compose easily}:
a circuit composed of multiple combinational devices connected with wires is also combinational
if each input of the constituent devices is tied to a single output or a constant value,
and if the network is \emph{acyclic}, i.e., it has no directed cycles
(which may create feedback loops).

Since Minispec functions are acyclic, one can synthesize a function by synthesizing each syntax element
(expressions, statements, etc.) and connecting them.
Though the details of synthesis depend on the compiler and tools used,
it is useful to roughly understand how different syntax elements are translated and composed.
Specifically:
\begin{compactitem}
% Expressions
\item Each complex expression is broken down into a tree of basic operations (e.g., operators or function calls), then each operation is synthesized,
  and finally their inputs and outputs are wired together in the same tree fashion.
\item All operators (\autoref{sec:expressions}) can be implemented using combinational circuits.
  Each operator is synthesized as a separate circuit.
\item Conditional expressions translate to multiplexers: each ternary operator is a 2-to-1 multiplexer, and each case expression is an N-to-1 multiplexer.
\item Concatenation expressions, struct creation expressions, and selection expressions that use statically known indices require wires but no logic:
  they are just combining or selecting bits of existing variables.
  Selection expressions that use dynamic indices require some logic (e.g., a barrel shifter).
\item Functions are \emph{inlined}: each function call instantiates a separate circuit implementing the function, which is then synthesized.
  (since there's no recursion, inlining always terminates).
% Statements
\item Variable assignments require no logic. Each assignment is simply naming the value (i.e., wires) of a particular expression,
  so it can be used somewhere else.
\item If statements are translated to multiplexers: all expressions within the if statement are synthesized,
  then each variable assigned to within the if statement is followed by a multiplexer
  to select between the value assigned within the if branch (if the predicate is true)
  and the value before the if branch (if the predicate is false).
  If-else statements are similar: the if and else branches are both synthesized,
  and the multiplexer selects between the value in the if branch and the value in the else branch.
\item Case statements are similar to if statements, requiring a multiplexer for each value assigned
  within the statement to select the right branch.
\item For loops are \emph{unrolled}: Each iteration is expanded into a separate block, then synthesized as usual.
\item A single return statement at the end of the function simply names the output of the function and requires no logic.
  In functions with multiple return statements, the output value is selected using multiplexers similar to if-else statements.
\end{compactitem}

\todo{Figures showing the synthesis of several snippets.}


