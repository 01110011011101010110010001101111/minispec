\section{Statements}
\label{sec:statements}

Statements, such as variable declarations or assignments, are the basic syntax unit of Minispec logic.
For example, each function (\autoref{sec:functions}) consists of a sequence of statements separated by semicolons.
Each statement expresses an action to be carried out.
Whereas expressions \emph{always} evaluate to a value and \emph{never} have side effects,
statements \emph{never} evaluate to a value and \emph{often} have side effects.
Semantically, statements appear to be evaluated in sequence, which makes code easy to understand.
% TODO: Not sure this is clear, but it's important. There's an option to pull synthesis details to this section, but I'd avoid it...
However, sequences of statements are synthesized into parallel combinational logic.
\autoref{sec:combinational} explains how this synthesis is done.

\begin{wrapfigure}{r}{0.32\columnwidth}
\vspace{-3.5em}
\begin{mscode}
// Variable declarations
//  Without initialization
Bit#(1) a;
//  With initialization
Bit#(2) b = 2'b11;
//  Using let
let c = 3'b011; // Bit#(3)
//  Multiple variables
Bool d = True, e = False;

// Assignments
a = 1'b1;
b[1] = 1'b0;
c[2:1] = b + 1;
typedef struct {
  Bool m; Bit#(2) n;
} ExStruct;
ExStruct s;
s.n[1] = 0;

// Lexical scoping
if (d) begin
  let x = !e;
  $display(x); // in scope
end
$display(x);
// Error, x out of scope

// Shadowing
if (d) begin
   let d = !e; // shadows
   $display(d);
end
$display(d); // OK, uses
             // outer d
\end{mscode}
\vspace{-9.5em}
\end{wrapfigure}

\subsection{Variable statements}
\label{sec:varStmts}

Variables are names for intermediate values.

\paragraph{Variable declaration and initialization:} The basic variable declaration statement has the form:
\begin{center}
\verb|VarType varName;|
\end{center}
where \verb|VarType| is the variable's type and \verb|varName| is its name.
Variables declared this way are \emph{uninitialized}, and cannot be used until they are assigned a value.

Variables can be declared and initialized in the same statement:
\begin{center}
\verb|VarType varName = initExpr;|
\end{center}
where, \verb|initExpr| is an expression compatible with type \verb|VarType|.

When declaring and initializing a variable, the \verb|let| keyword can be used in place of \verb|VarType|:
\begin{center}
\verb|let varName = initExpr;|
\end{center}
In this case, the compiler will infer the variable's type from \verb|initialExpr|'s type.

Finally, multiple variables of the same type can be declared and optionally initialized in the same statement:
\begin{center}
\verb|VarType varName1 = initlExpr1, ..., varNameN = initlExprN;|
\end{center}

\paragraph{Variable assignment:} Each assignment statement binds a variable to a value. Its syntax is %:
%\begin{center}
\verb|lValue = expr;|,
%\end{center}
where \verb|lValue| can be \emph{(i)} variable name,
\emph{(ii)} a struct member or submodule input (e.g., \verb|structVarName.memberName|),
\emph{(iii)} a single bit or vector element (e.g., \verb|varName[bitNum]|),
\emph{(iv)} a range of a bits from a \verb|Bit#(n)| variable (e.g., \verb|varName[hi:lo]|),
or \emph{(v)} a nested combination of any of the above.

A variable can be assigned to multiple times. Each assignment changes the value bound to the variable.
Every statement following the assignment sees the new value.

Variables of compound types (i.e., structs, vectors, and \verb|Bit#(n)|)
can be declared uninitialized and then initialized element by element.
However, it is illegal to use a \verb|Bit#(n)| variable that is partially initialized,
even if the particular bits being accessed have been initialized.
To avoid this problem, you can initialize the entire \verb|Bit#(n)| variable
(e.g., to \verb|0| or \verb|?|), then reassign the individual bits.
Structs and vectors do not suffer from this limitation.

\paragraph{Variables are lexically scoped:}
Like most languages, Minispec uses \emph{lexical scoping}: a variable may only be used inside the block of code it is defined.
For example, a variable declared within a begin-end block cannot be used outside the block.

\paragraph{Name clashes and variable shadowing:}
It is illegal to declare two or more variables with the same name in the same scope (i.e., block of code).
Scopes can be nested, and it is allowed, \emph{though not recommended},
to declare a variable in a nested scope with the same name as another variable in a surrounding scope.
In this case, we say the variable in the inner scope \emph{shadows} (i.e., hides) the variable in the outer scope.
Shadowing can be confusing and error-prone, so the compiler emits warnings on every shadowed variable.

\subsection{Begin-end statements}

A begin-end statement denotes a block of code. It allows combining multiple statements into a single statement.
Begin-end blocks can be used anywhere a statement is required, and are often used with control-flow
statements \verb|if| and \verb|for|. Its syntax is:
\begin{center}
\verb|begin stmt1 ... stmtN end|
\end{center}
where \verb|stmt|$_{\texttt{i}}$ are statements. Each begin-end block initiates a new lexical context,
which supports local variable declarations.

Note that \emph{begin-end statements are not terminated by semicolons},
and placing a \verb|;| after the \verb|end| keyword will cause an error.
This follows BSV and SystemVerilog/Verilog syntax.
In general, keywords that start with \verb|end| (e.g., \verb|end|, \verb|endfunction|, \verb|endcase|, etc.)
are not followed by \verb|;|, while all other statements are terminated by \verb|;|.

\begin{wrapfigure}{r}{0.32\columnwidth}
\vspace{-4.5em}
\begin{mscode}
Bit#(2) a = 2'd2;
Bool b = True;

// If statements
Bit#(8) x = 0;
if (a > 2) x = {0, a};

Bit#(4) y;
if (b) y = x;
else y = ~x;

Bit#(2) z;
if (a > 2) z = x;
else if (b) z = y;
else begin
  let w = foo(y);
  z = w + 1;
end

// Case statement
case (z):
  1 : x = 1;
  2 : begin
        x = 2;
        z = y + 1;
      end
  default : x = 0;
endcase

// For loop
Bit#(6) w;
for (Integer i=0; i<6; i=i+1)
  w[i] = x[i % 2];
\end{mscode}
\vspace{-7.5em}
\end{wrapfigure}

% dsm; Deferred
%// Return statement
%function Bit#(2) f(Bit#(4) a);
%  return truncate(a + 3);
%endfunction

\subsection{Control-flow statements}

Minispec supports conditional statements and (restricted) loops.
Though similar in syntax to those of other programming languages,
these constructs synthesize to combinational logic.
Here we explain the syntax and semantics of each statement,
and \autoref{sec:combinational} explains their synthesis to hardware.

\paragraph{If statements} have the following syntax:
\begin{center}
\verb|if (condExpr) trueStmt |\optStart\verb| else falseStmt |\optEnd
\end{center}
where \verb|condExpr| is a \verb|Bool| expression, and \verb|trueStmt| and \verb|falseStmt| are statements.
If \verb|condExpr| evaluates to \verb|True|, then \verb|trueStmt| is executed.
Otherwise, if the optional else clause is present, \verb|falseStmt| is executed.

As shown in the examples, if statements often use begin-end blocks, and
multiple if-else statements can be chained (\verb|if (cond1) ... else if (cond2) ...|).

\paragraph{Case statements} have a similar syntax to case \emph{expressions} (\autoref{sec:condExprs}):
\begin{center}
\verb|case (compExpr)           | \\
\verb|  value1 : stmt1;         | \\
\verb|  value2 : stmt2;         | \\
\verb|        ...               | \\
\verb| | \optStart \verb| default : defaultStmt; |\optEnd \\
\verb|endcase                   |
\end{center}
A case statement tests \verb|compExpr| against the \verb|value|$_{\texttt{i}}$ values,
and on a match with \verb|value|$_{\texttt{i}}$, \verb|stmt|$_{\texttt{i}}$ is executed.
If there are no matches and the optional default label is specified, \verb|defaultStmt| is executed.
Unlike case expressions, case statements need not enumerate all values or specify a default statement.
If there are no matches and no default, none of the statements is executed.

\paragraph{For loop statements} allow compactly expressing a sequence of similar statements. They have the usual syntax:
\begin{center}
\verb|for (Integer iVar = initExpr; testExpr; iVar = updExpr) stmt;|\
\end{center}
where \verb|iVar| is the name of the \verb|Integer| induction variable;
\verb|initExpr| is the induction variable's initial value;
\verb|testExpr| is a \verb|Bool| expression that denotes whether to stop iterations;
\verb|updExpr| is evaluated after each iteration to update \verb|iVar|;
and \verb|stmt| is the statement executed on each iteration.

For loops are \emph{not} like general loops in programming languages.
Whereas general loops may iterate on values unknown at compile time and may have an unknown number of iterations,
Minispec for loops \emph{have a known iteration count} and \emph{are unrolled at compile time}.
This makes for loops implementable with combinational logic.
To guarantee that the iteration count is known, the induction variable is always an \verb|Integer|.
For example, the loop in the example to the right is unrolled into:

\begin{mscode}
  w[0] = z[0];  w[1] = z[1];  w[2] = z[0];  w[3] = z[1];  w[4] = z[0];  w[5] = z[1];
\end{mscode}

\paragraph{Return statements} are used to return a value from a function or method. Their syntax is simply \verb|return expr;|.
\autoref{sec:functions} explains where they can appear and provides further examples.

\subsection{Register-write statements}

A register-write statement performs a write to a register module.
Its syntax is \verb|regName <= expr;|.
Registers are the basic building block of modules and sequential logic,
so \autoref{sec:modules} details the semantics and restrictions of register writes.
