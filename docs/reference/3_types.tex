\section{Overview of Types}
\label{sec:types}

Minispec is a \emph{strongly typed} language with \emph{static type checking}:
every variable and expression has a \emph{type},
and that type must be known statically, i.e., at compile time.
Variables must be assigned values with compatible types.

\paragraph{Built-in types:} Minispec provides five basic, built-in types:
\begin{compactitem}
\item \texttt{Bool} represents a Boolean value, which can be either \texttt{True} or \texttt{False}.
\item \texttt{Bit\#(n)} represents an n-bit value.
\item \texttt{Vector\#(n, T)} represents a collection of n values of type T.
\item \texttt{Maybe\#(T)} represents an optional value of type T.
\item \texttt{Integer} represents an integer value with an unbounded number of bits.
\end{compactitem}
\autoref{sec:builtin} describes and illustrates the use of these types in detail.
All types except \texttt{Integer} are ultimately represented as a collection of bits in hardware,
and can be converted to and from \texttt{Bit\#(n)} (\autoref{sec:bit}).
\texttt{Integer} is special: it cannot be synthesized into hardware
%(it has an unbounded number of bits!)
and can only be used at compile time.

\paragraph{User-defined types:} Minispec supports three kinds of user-defined types:
\begin{compactitem}
\item \emph{Type synonyms} allow giving a different name to a type.
\item \emph{Structs} represent collections of values.
\item \emph{Enumerations} or enums allow defining a set of unique symbolic constants, also called labels.
\end{compactitem}
\autoref{sec:userdefined} describes user-defined types in detail.

\begin{wrapfigure}{r}{0.32\columnwidth}
\vspace{-2em}
\begin{mscode}
// A 16-bit value
Bit#(16) halfWord;

// A Vector of 8 Bools
Vector#(8, Bool) boolVec;

// A Vector of 16 Bit#(32)s
Vector#(16,Bit#(32)) wordVec;
\end{mscode}
\vspace{-2em}
\end{wrapfigure}

\paragraph{Parametric types:} Types of the form \texttt{Type\#($\rho_1$,...,$\rho_n$)} are called parametric types
(also known as polymorphic types or generics in other languages).
Parametric types take one or more parameters ($\rho_i$), which can be either \texttt{Integer} values or other types.
These parameters must be known at compile time, and make the type concrete.
Several of the built-in types are parametric, as shown in the examples.

Beyond parametric types, Minispec also supports parametric functions, modules, structs, and type synonyms
using a consistent syntax and semantics, which \autoref{sec:parametrics} describes in detail.

\paragraph{Type conversions are explicit, except for Integers:}
Unlike in some languages, in Minispec almost all conversions between values of different types are explicit.
For example, it is illegal to assign a \texttt{Bool} value to a \texttt{Bit\#(1)} variable (or viceversa),
even though both take one bit to represent.
Similarly, it is illegal to assign \texttt{Bit\#(4)} value to a \texttt{Bit\#(8)} variable, because they have a different number of bits
and implicit bit-width extensions \emph{never} happen.
This makes code more verbose but reduces mistakes.

\begin{mscode}
Bool b = True;
Bit#(1) x = b;         // Error, cannot convert from Bool to Bit#(1)
Bit#(1) y = b? 1 : 0;  // OK, uses ternary operator to convert explicitly

Bit#(4) i = 12;
Bit#(8) j = i;         // Error, mismatched bit-widths 4 and 8
Bit#(8) k = {0, i};    // OK, uses concatenation to zero-extend i
\end{mscode}

\begin{wrapfigure}{r}{0.32\columnwidth}
\vspace{-1.8em}
\begin{mscode}
Integer n = 3 * 4;  // n=12
Bit#(8) x = n;      // OK
Bit#(n) y = n * n;  // OK
\end{mscode}
\vspace{-2em}
\end{wrapfigure}

% TODO: May want to move to parametrics
\texttt{Integer} values are the single exception to this rule:
\texttt{Integer} values can be assigned to \texttt{Bit\#(n)} variables without explicit conversion.
In fact, \texttt{Integer}s can be used anywhere a unsized number literal would work,
as shown in the examples.

%An intuitive way to remember this exception is to think of the compiler as evaluating all \texttt{Integer} expressions,
%computing their corresponding unsized number literals.

\begin{wrapfigure}{r}{0.32\columnwidth}
\vspace{-1.5em}
\begin{mscode}
Bit#(4) x = 1;   
let y = x;       // Bit#(4)
let z = {x, x};  // Bit#(8)
let w = 2'b11;   // Bit#(2)
let n = 42;      // Integer
\end{mscode}
\vspace{-3em}
\end{wrapfigure}

% dsm: Eh, whatever
%// Error, can't infer a's
%// width because 0 is unsized
%let a = {0, x}

\paragraph{Type inference:} Minispec allows omitting a variable's type by using the \texttt{let} keyword.
The compiler will infer the variable's type from the expression assigned to the variable.

% dsm: Yes, this goes into expressions
%(TODO: expresssions??)
%Built-in operators:
%- Boolean
%- Logical
%- Arithmetic
%- Relational
%- Ternary
%- Case
%- Slicing and concatenation
%- Structs?
%- Calls

%Working with Bit(n)

